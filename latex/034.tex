\chapter{Coordination Problems, Part Ⅱ}

Minerva and Dumbledore together had applied their combined talent to
conjure the grand stage toward which Quirrell now slowly trudged; it
was, at its core, sturdy wood, but the outer surfaces shone with glitter
of marble inlaid with platinum and studded with gems of every House
colour. Neither she nor the Headmaster was any Founder of Hogwarts, but
the conjuration only needed to last a few hours. Minerva ordinarily
enjoyed the few occasions when she had the occasion to tire herself out
on large Transfigurations; she should have enjoyed the many small
chances for artistry, and the illusion of opulence; but this time she
had done the work with the dreadful feeling of digging her own grave.

But Minerva was feeling a little better now. There'd been one brief
moment when the explosion might've come; but Dumbledore had already been
standing up and applauding warmly, and no one had proven foolish enough
to riot in front of the Headmaster.

And the explosive mood had rapidly faded into a collective sentiment
which might perhaps have been described by the phrase: \emph{Give us a
break!}

Blaise Zabini had shot himself in the name of Sunshine, and the final
score had been 254 to 254 to 254.

\begin{center}\rule{3in}{0.4pt}\end{center}

Behind the stage, waiting to ascend, three children were glaring at each
other in mingled fury and frustration. It didn't help that they were
still damp from being fished out of the lake, and that the Warming
Charms didn't seem quite enough to make up for the crisp December air,
or maybe it was just their mood.

``That's \emph{it},'' said Granger. ``I've \emph{had} it! No more
traitors!''

``I completely agree with you, Miss Granger,'' Draco said icily.
``Enough is enough.''

``And what do \emph{you} two intend to do about it?'' snapped Harry
Potter. ``Professor Quirrell already said he wouldn't ban spies!''

``We'll ban them \emph{for} him,'' said Draco grimly. He hadn't even
understood what he meant by the words as he said them, but the very act
of speaking seemed to crystallise a plan---

\begin{center}\rule{3in}{0.4pt}\end{center}

The stage really was well done, at least for a temporary structure; the
makers hadn't fallen into the usual pitfall of being impressed by their
own illusion of wealth, and knew something about architecture and visual
style. From where Draco stood, in the obvious place for him to stand,
the watching students would see him haloed in the faint glitter of
emeralds; and Granger, standing where Draco had subtly motioned her,
would be haloed in Ravenclaw's sapphire. As for Harry Potter, Draco
wasn't looking at him right now.

Professor Quirrell had\ldots{} awakened, or whatever it was he did; and
was leaning upon a platinum podium bare of all gems. With evident
showmanship, the Defence Professor was carefully stacking and squaring
those three envelopes containing the three parchments upon which the
three generals had written their wishes, as all the students of Hogwarts
watched, and waited.

Finally Professor Quirrell looked up from the envelopes. ``Well,'' said
the Defence Professor. ``This is inconvenient.''

A slight titter of laughter ran through the crowd, with a sharp
undertone.

``I suppose you are all wondering what I will do?'' said Professor
Quirrell. ``There is nothing for it; I shall have to do what is fair.
Although first there was a little speech I wanted to make, and before
even that, it appears to me that Mr.~Malfoy and Miss Granger have
something they wish to share.''

Draco blinked, and then he and Granger traded rapid glances---\emph{may
I?}---\emph{yes, go ahead}---and Draco raised his voice.

``General Granger and I would both like to say,'' Draco said in his most
formal voice, knowing it was being amplified and heard, ``that we will
no longer accept the help of any traitors. And if, in any battle, we
find that Potter has accepted traitors from either of our armies, we
will join forces to crush him.''

And Draco shot a glance filled at malice at the Boy-Who-Lived.
\emph{Take that, General Chaos!}

``I agree completely with General Malfoy,'' said Granger standing beside
him, her high voice clear and strong. ``Neither of us will use traitors,
and if General Potter does, we will wipe him off the battlefield.''

There was a susurration of surprise from the watching students.

``Very good,'' said their Defence Professor, smiling. ``It took the two
of you long enough, but you are still to be congratulated on having
thought of it before any other generals.''

It took a moment for this to soak in---

``In the future, Mr.~Malfoy, Miss Granger, before you come to my office
with any request, consider whether there is a way for you to accomplish
it without my help. I will not deduct Quirrell points on this occasion,
but next time you may expect to lose the full fifty.'' Professor
Quirrell wore an amused grin. ``And what do you have to say about that,
Mr.~Potter?''

Harry Potter's gaze went to Granger, then to Draco. His face appeared
calm; though Draco was sure \emph{controlled} would have been the better
term.

Finally Harry Potter spoke, his voice level. ``The Chaos Legion is still
happy to accept traitors. See you on the battlefield.''

Draco knew the shock was showing on his own face; there were astonished
murmurs from the watching students, and when Draco glanced at the front
row he saw that even Harry's Chaotics looked taken aback.

Granger's face was angry, and getting angrier. ``Mr.~Potter,'' she said
in a sharp tone like she thought she was a teacher, ``are you
\emph{trying} to be obnoxious?''

``Most certainly not,'' Harry Potter said calmly. ``I won't make you do
it every time. Beat me once, and I'll stay beaten. But threats aren't
always enough, General of Sunshine. You did not ask me to join with you,
but tried simply to impose your will; and sometimes you must actually
defeat the enemy, to impose your will on him. You see, I am skeptical
that Hermione Granger, the brightest academic star of Hogwarts, and
Draco, son of Lucius, scion of the Noble and Most Ancient House of
Malfoy, can work together to beat their common foe, Harry Potter.'' An
amused smile crossed Harry Potter's face. ``Maybe I'll just do what
Draco tried with Zabini, and write a letter to Lucius Malfoy and see
what \emph{he} thinks about that.''

\emph{``Harry!''} gasped Granger, looking absolutely aghast, and there
were gasps from the audience as well.

Draco controlled the anger flushing through him. That had been a
\emph{stupid} move on Harry's part, saying that in public. If Harry had
simply \emph{done} it, it might have worked, Draco hadn't even thought
about that, but \emph{now} if Father did that it would look like he was
playing into Harry's hands---

``If you think my father, Lord Malfoy, can be manipulated by \emph{you}
that easily,'' Draco said coldly, ``you have a surprise coming, Harry
Potter.''

And Draco realised as the words finished leaving his mouth that he'd
just backed \emph{his own father} squarely into the corner, more or less
without even meaning to. Father probably \emph{wasn't} going to like
this, not the tiniest bit, but now it would be impossible for him to say
so\ldots{} Draco would have to apologise for that, it \emph{had} been an
honest accident, but it was strange to think that he'd done it at all.

``Then go ahead and defeat the evil General Chaos,'' Harry said, still
looking amused. ``I can't win against both your armies---not if you
\emph{really} work together. But I wonder if perhaps I could break you
up before then.''

``You won't, and we'll \emph{crush} you!'' said Draco Malfoy.

And beside him, Hermione Granger firmly nodded.

``Well,'' said Professor Quirrell after the astounded silence had
stretched for a while. ``That was \emph{not} how I expected that
particular conversation to go.'' The Defence Professor had a rather
intrigued expression on his face. ``Truthfully, Mr.~Potter, I expected
you to concede immediately and with a smile, then announce that you had
long since worked out my intended lesson but had decided not to spoil it
for others. Indeed, I planned my speech accordingly, Mr.~Potter.''

Harry Potter just shrugged. ``Sorry about that,'' he said, and said
nothing more.

``Oh, don't worry,'' said Professor Quirrell. ``This, too, will serve.''

And Professor Quirrell turned from the three children, and straightened
at the podium to address the whole watching crowd; his customary air of
detached amusement dropped away like a falling mask, and when he spoke
again his voice was amplified louder than it had been.

``If not for Harry Potter,'' said Professor Quirrell, his voice as crisp
and cold as December, ``You-Know-Who would have won.''

The silence was instant, and total.

\begin{center}\rule{3in}{0.4pt}\end{center}

``Make no mistake,'' said Professor Quirrell. ``The Dark Lord \emph{was}
winning. There were fewer and fewer Aurors who dared face him, the
vigilantes who opposed him were being hunted down. One Dark Lord and
perhaps fifty Death Eaters were \emph{winning} against a country of
thousands. That is beyond ridiculous! There are no grades low enough for
me to mark that incompetence!''

There was a frown on the face of Headmaster Dumbledore; and on the faces
of the audience, puzzlement; and the utter silence went on.

``Do you understand now how it happened? You saw it today. I allowed
traitors, and gave the generals no means to restrain them. You saw the
result. Clever plots and clever betrayals, until the last soldier left
on the battlefield shot himself! You cannot \emph{possibly} doubt that
all three of those armies could have been defeated by \emph{any} outside
foe that was unified within itself.''

Professor Quirrell leaned forward at the podium, his voice now filled
with a grim intensity. His right hand stretched out, fingers open and
spread. ``Division is weakness,'' said the Defence Professor. His hand
closed into a tight fist. ``Unity is strength. The Dark Lord understood
that well, whatever his other follies; and he \emph{used} that
understanding to create the one simple invention that made him the most
terrible Dark Lord in history. Your parents faced one Dark Lord. And
fifty Death Eaters who were perfectly unified, knowing that any breach
of their loyalty would be punished by death, that any slack or
incompetence would be punished by pain. None could escape the Dark
Lord's grasp once they took his Mark. And the Death Eaters agreed to
take that terrible Mark because they knew that once they took it, they
would be \emph{united}, facing a divided land. One Dark Lord and fifty
Death Eaters would have defeated an entire country, by the power of the
Dark Mark.''

Professor Quirrell's voice was bleak and hard. ``Your parents
\emph{could} have fought back in kind. They did not. There was a man
named Yermy Wibble who called upon the nation to institute a draft,
though he did not quite have vision enough to propose a Mark of Britain.
Yermy Wibble knew what would happen to him; he hoped his death would
inspire others. So the Dark Lord took his family for good measure. Their
empty skins inspired nothing but fear, and no one dared to speak again.
And your parents would have faced the consequences of their despicable
cowardice, if not for being saved by a one-year-old boy.'' Professor
Quirrell's face showed full contempt. ``A dramatist would have called
that a \emph{dei ex machina,} for they did nothing to earn their
salvation. He-Who-Must-Not-Be-Named may not have deserved to win, but
make no doubt of it, your parents deserved to lose.''

The voice of the Defence Professor rang forth like iron. ``And know
this: your parents have learned nothing! The nation is still fragmented
and weak! How few decades passed between Grindelwald and You-Know-Who?
Do you think \emph{you} will not see the next threat in your own
lifetimes? Will \emph{you} repeat then the follies of your parents, when
you have seen the results so clearly laid out before you this day? For I
can tell you what your parents will do, when the day of darkness comes!
I can tell you what lesson they have learned! They have learned to hide
like cowards and do nothing while they wait for Harry Potter to save
them!''

There was a wondering look in the eyes of Headmaster Dumbledore; and
other students gazed up at their Defence Professor with bewilderment and
anger and awe.

Professor Quirrell's eyes were as cold now as his voice. ``Mark this,
and mark it well. He-Who-Must-Not-Be-Named wished to rule over this
country, to hold it in his cruel grasp forever. But at least he wished
to rule over a \emph{living} country, and not a heap of ash! There have
been Dark Lords who were mad, who wished only to make the world a vast
funeral pyre! There have been wars in which one whole country marched
against another! Your parents nearly lost against half a hundred, who
thought to take this country alive! How quickly would they have been
crushed by a foe more numerous than they, a foe that cared for nothing
but their destruction? This I foretell: When the next threat rises,
Lucius Malfoy will claim that you must follow him or perish, that your
only hope is to trust in his cruelty and strength. And though Lucius
Malfoy himself will believe it, this will be a lie. For when the Dark
Lord perished, Lucius Malfoy did not unite the Death Eaters, they were
shattered in an instant, they fled like whipped dogs and betrayed each
other! Lucius Malfoy is not strong enough to be a true Lord, Dark or
otherwise.''

Draco Malfoy's fists were clenched white, there were tears in his eyes,
and fury, and unbearable shame.

``No,'' said Professor Quirrell, ``I do not think it will be Lucius
Malfoy who saves you. And lest you think that I speak on my own behalf,
time will make clear soon enough that this is not so. I make you no
recommendation, my students. But I say that if a whole country were to
find a leader as strong as the Dark Lord, but honorable and pure, and
take his Mark; then they could crush any Dark Lord like an insect, and
all the rest of our divided magical world could not threaten them. And
if some still greater enemy rose against us in a war of extermination,
then only a united magical world could survive.''

There were gasps, mostly from Muggleborns; the students in green-trimmed
robes looked merely puzzled. Now it was Harry Potter whose fists were
clenched tight and trembling; and Hermione Granger beside him was angry
and dismayed.

The Headmaster rose from his seat, his face now stern, saying no word as
yet; but the command was clear.

``I do not say \emph{what} threat will come,'' said Professor Quirrell.
``But you will not live all your lives in peace, not if the past history
of the world is any guide at all to its future. And if you do in the
future as you have seen three armies do this day, if you cannot throw
aside your petty bickering and take the Mark of a single leader, then
indeed you might wish that the Dark Lord had lived to rule over you, and
regret the day that ever Harry Potter was born---''

\emph{``Enough!''} bellowed Albus Dumbledore.

There was silence.

Professor Quirrell slowly turned his head to gaze at where Albus
Dumbledore stood in the fury of his wizardry; their eyes met, and a
soundless stress pressed down like weight upon all the students, as they
listened not daring to move.

``You, too, failed this country,'' said Professor Quirrell. ``And you
know the peril as well as I.''

``Such speeches are not for the ears of students,'' said Albus
Dumbledore in a dangerously rising voice. ``Nor for the mouths of
professors!''

Dryly, then, Professor Quirrell spoke: ``There were many speeches made
for the ears of adults, as the Dark Lord rose. And the adults clapped
and cheered, and went home having enjoyed their day's entertainment. But
I will obey you, Headmaster, and make no further speeches if you do not
like them. My lesson is simple. I will go on doing nothing about
traitors, and we will see what students can do for themselves about
that, when they do not wait for professors to save them.''

And then Professor Quirrell turned back to his students, and his mouth
quirked up in a wry grin that seemed to dissipate the dreadful pressure
like a god blowing to scatter the clouds. ``But do please be kind to the
traitors up until now,'' said Professor Quirrell. ``They were just
having fun.''

There was laughter, though it was nervous at first, and then it seemed
to build, as Professor Quirrell stood there smiling wryly and some of
the tension released itself.

\begin{center}\rule{3in}{0.4pt}\end{center}

Draco's mind was still whirling through a thousand questions and a daze
of horror, as Professor Quirrell prepared to open the envelopes in which
the three had inscribed their wishes.

It had never before occurred to Draco that moon-traveling Muggles were a
greater threat than the slow decline of wizardry, or that Father had
proven himself too weak to stop them.

And even stranger, the obvious implication: Professor Quirrell believed
that \emph{Harry} could. The Defence Professor claimed to have made no
recommendation, but he'd mentioned Harry Potter over and over in his
speech; others would already be thinking the same thing as Draco.

It was ridiculous. The boy who had covered a stuffed chair in glitter
and called it a throne---

\emph{The boy who faced down Snape and won,} whispered a traitorous
voice, \emph{that boy could grow into a Lord strong enough to rule,
strong enough to save us all---}

Harry had been \emph{raised} by Muggles! He was practically a mudblood
himself, he wouldn't fight against his adopted family---

\emph{He knows their arts, their secrets and their methods; he can take
all of the Muggles' science and use it against them, alongside our own
power as wizards.}

But what if he refuses? What if he's too weak?

\emph{Then,} said that inner voice, \emph{it will have to be you, won't
it, Draco Malfoy?}

And then there was a renewed hush from the crowd, as Professor Quirrell
opened the first envelope.

``Mr.~Malfoy,'' said Professor Quirrell, ``your wish is for\ldots{}
Slytherin to win the House Cup.''

There was a puzzled pause from the watching audience.

``Yes, Professor,'' said Draco in a clear voice, knowing that it was
once again being amplified. ``If you can't do that, then something else
for Slytherin---''

``I will not award House points unfairly,'' said Professor Quirrell. He
tapped a cheek, looking thoughtful. ``Which makes your wish difficult
enough to be interesting. Would you like to say anything about why,
Mr.~Malfoy?''

Draco turned from the Defence Professor, gazed out at the crowd from
against that backdrop of platinum and emeralds. Not all of Slytherin had
cheered for Dragon Army, there were anti-Malfoy factions who had
expressed that dissatisfaction by supporting the Boy-Who-Lived, or even
Granger; and those factions would be encouraged greatly by what Zabini
had done. He needed to remind them that Slytherin meant Malfoy and
Malfoy meant Slytherin---

``No,'' said Draco. ``They're Slytherins, they'll understand.''

There was some laughter from the audience, especially in Slytherin, even
from some students who would have called themselves anti-Malfoy a moment
earlier.

Flattery was a lovely thing.

Draco turned back to look at Professor Quirrell again, and was surprised
to see an embarrassed look on Granger's face.

``And for Miss Granger\ldots{}'' said Professor Quirrell. There was the
sound of a tearing envelope. ``Your wish is for\ldots{} Ravenclaw to win
the House Cup?''

There was considerable laughter from the audience, including a chuckle
from Draco. He hadn't thought Granger played that game.

``Well, um,'' said Granger, sounding like she was suddenly stumbling
over a memorised speech, ``I meant to say, that\ldots{}'' She took a
deep breath. ``There were soldiers from every House in my army, and I
don't mean to slight any of them. But Houses should still count for
something, too. It was sad when students in the same House were hexing
each other just because they were in different armies. People should be
able to rely on whoever's in their House. That's why Godric Gryffindor,
and Salazar Slytherin, and Rowena Ravenclaw, and Helga Hufflepuff
created the four Houses of Hogwarts in the first place. I'm the General
of Sunshine, but even before that, I'm Hermione Granger of Ravenclaw,
and I'm proud to be part of a House that's eight hundred years old.''

``Well said, Miss Granger!'' said Dumbledore's booming voice.

Harry Potter was frowning, and something tickled at the edge of Draco's
recognition.

``An interesting sentiment, Miss Granger,'' said Professor Quirrell.
``But there are times when it is good for a Slytherin to have friends in
Ravenclaw, or for a Gryffindor to have friends in Hufflepuff. Surely it
would be best if you could rely both on your friends in your House, and
also your friends in your army?''

Granger's eyes flicked briefly toward the watching students and
teachers, and she said nothing.

Professor Quirrell nodded as though to himself, and then turned back to
the podium, and took up and tore open the last envelope. Beside Draco,
Harry Potter visibly tensed up as the Defence Professor drew forth the
parchment. ``And Mr.~Potter wishes for---''

There was a pause as Professor Quirrell looked at the parchment.

Then, without any change of expression on Professor Quirrell's face, the
sheet of parchment burst into flames, and burned with a brief, intense
fire that left only drifting black dust sprinkling down from his hand.

``Please confine yourself to the possible, Mr.~Potter,'' said Professor
Quirrell, sounding very dry indeed.

There was a long pause; Harry, standing beside Draco, looked rather
shaken.

\emph{What in Merlin's name did he ask for?}

``I do hope,'' said Professor Quirrell, ``that you prepared another
wish, if I could not grant that one.''

There was another pause.

Harry drew a deep breath. ``I didn't,'' he said, ``but I already thought
of another one.'' Harry Potter turned to look out at the audience, and
his voice firmed as he spoke. ``People fear traitors because of the
damage the traitor does directly, the soldiers they shoot or the secrets
they tell. But that's only part of the danger. What people do because
they're \emph{afraid} of traitors also costs them. I used that strategy
today against Sunshine and Dragon. I didn't tell my traitors to cause as
much direct damage as possible. I told them to act in the way that would
create the most distrust and confusion, and make the generals do the
most costly things to try and stop them from doing it again. When there
are just a few traitors and a whole country opposing them, it stands to
reason that what a few traitors do might be less damaging than what a
whole country does to stop them, that the cure might be worse than the
disease---''

``Mr.~Potter,'' said the Defence Professor, his voice suddenly cutting,
``the lesson of history is that you are simply wrong. Your parents'
generation did too little to unify themselves, not too much! This whole
country almost fell, Mr.~Potter, though you were not there to see it. I
suggest that you ask your dorm-mates in Ravenclaw how many of them have
lost family to the Dark Lord. Or if you are wiser, do \emph{not} ask!
\emph{Do} you have a wish to make, Mr.~Potter?''

``If you don't mind,'' said the mild voice of Albus Dumbledore, ``I
should like to hear what the Boy-Who-Lived has to say. He has more
experience than either of us at stopping wars.''

A few people laughed, but not many.

Harry Potter's gaze moved to Dumbledore, and he looked considering for a
moment. ``I'm not saying you're wrong, Professor Quirrell. In the last
war, people didn't act together, and a whole country almost fell to a
few dozen attackers, and yes, that was pathetic. And if we make the same
mistake next time, yes, that'll be even more pathetic. But you never
fight the same war twice. And the problem is, the enemy is \emph{also}
allowed to be smart. If you're divided you're vulnerable in one way; but
when you try to unite, then you face other risks, and other costs, and
the enemy will try to take advantage of those, too. You can't stop
thinking at just one level of the game.''

``Simplicity also has a great deal to commend it, Mr.~Potter,'' said the
dry voice of the Defence Professor. ``I do hope that you have learned
something this day about the dangers of strategies more complicated than
uniting your people and attacking your enemy. And if all this does not
tie into your wish somehow, I shall be quite annoyed.''

``Yes,'' said Harry Potter, ``it was pretty difficult coming up with a
wish to symbolise the costs of unity. But the problem of acting together
isn't just for wars, it's something we have to solve all our lives,
every day. If everyone is coordinating using the same rules, and the
rules are stupid, then if \emph{one} person decides to do things
differently, they're breaking the rules. But if \emph{everyone} decides
to do things differently, they can. It's exactly the same problem of
everyone needing to act together. But for the \emph{first} person who
speaks out, it seems like they're going against the crowd. And if you
thought that the only important thing was that people should always be
unified, then you could never change the game, no matter how stupid the
rules. So my own wish, to symbolise what happens when people unite in
the wrong direction, is that in Hogwarts we should play Quidditch
without the Snitch.''

\emph{``WHAT?''} screamed a hundred voices in the crowd, as Draco's jaw
dropped.

``The Snitch ruins the whole game,'' said Harry Potter. ``Everything the
other players do ends up being irrelevant. It would make overwhelmingly
more sense to just buy a clock. It's one of those incredibly stupid
things you don't notice just because you grew up with it, that people
only do because everyone else is doing it---''

But by that point Harry Potter's voice could no longer be heard, because
the riot had started.

\begin{center}\rule{3in}{0.4pt}\end{center}

The riot ended around fifteen seconds later, after a gigantic spout of
fire blasted out of the highest tower of Hogwarts to the sound of a
hundred thunders. Draco hadn't known Dumbledore could \emph{do} that.

The students sat down again very carefully and quietly.

Professor Quirrell was laughing, without pause. ``So be it, Mr.~Potter.
Your will be done.'' The Defence Professor paused deliberately. ``Of
course, I only promised \emph{one} cunning plot. And that is all that
the three of you will get.''

Draco had been half-expecting the words earlier, but now they still came
as a shock; Draco exchanged rapid glances with Granger, they would have
been the obvious allies but their wishes were directly opposed---

``You mean,'' said Harry, ``we have to all agree on a wish?''

``Oh, that would be \emph{far} too much to ask,'' said Professor
Quirrell. ``The three of you have no common enemy, do you?''

And for one brief moment, so fast that Draco thought he might have
imagined it, the Defence Professor's eyes flicked in the direction of
Dumbledore.

``No,'' said Professor Quirrell, ``I mean that I shall grant three
wishes using a single plot.''

There was a confused silence.

``You can't do that,'' Harry said flatly from beside Draco. ``Not even
\emph{I} can do that. Two of those wishes are mutually incompatible.
It's \emph{logically impossible---''} and then Harry cut himself off.

``You're a few years too young to tell me what I can't do, Mr.~Potter,''
said Professor Quirrell, with a brief dry smile.

Then the Defence Professor turned back to the watching students.
``Truthfully, I have no confidence in your ability to learn this day's
lesson. Go home, and enjoy your time with your families, or what's left
of them, while they still live. My own family is long since dead at the
Dark Lord's hand. I shall see you all when classes resume.''

In the speechless silence that resulted, Professor Quirrell already
turning to walk off the stage, Draco heard the Defence Professor's voice
say, quietly and no longer amplified, ``But you, Mr.~Potter, I would
speak to now.''
